% Preamble
\documentclass[11pt]{article}

% Packages
\usepackage{amsmath}
\usepackage{hyperref}

% sans-serif typeface
\renewcommand{\familydefault}{\sfdefault}

% hyperlink setup
\hypersetup{
    colorlinks=true,       % false: boxed links; true: colored links
    urlcolor=blue        % color of external links
}

% suppress page numbering
\pagenumbering{gobble}

% Document
\begin{document}
    \begin{flushleft}
        \LARGE \textbf{Distributional State Aggregation in Reinforcement Learning}
        \newline \break
        \Large \textbf{Shayan Amani}
        \newline \break
        \textit{MS Project Presentation}
        \newline \break

        Tuesday, August 25, 2020 at 10:00 AM EST \\
        via Zoom at \href{https://unh.zoom.us/j/98881910790}{https://unh.zoom.us/j/98881910790}

    \end{flushleft}
    \section*{Abstract}

    Despite the theoretical promises, real-world problems introduce computational challenges in the adaption of reinforcement learning algorithms.
    Such problems are often accompanied by an enormous state space, large enough to surpass the conventional memory and processing limits.
    State aggregation encompasses ideas and methods to downscale a state space with a bounded error to yield a compact instance of the original problem ultimately.
    Being around for decades, the recent advancement in state aggregation went in the shadow of approaches based on neural networks, though.
    As an analytically-transparent method, state aggregation still has superior advantages over incorporating neural networks. \\

    While a variety of approaches are available in the literature, the majority of the solutions are whether highly dependent on problem mechanics or make strong theoretical assumptions that render the results impractical.
    Universal state aggregation methods are deemed to be necessary regardless of the low- level assumptions and properties.
    This work proposes a class of non-parametric state aggregation methods merely based on the collected sample distribution.
    Inspired by histograms as the density estimator, a myriad of practices are investigated, then experiments are performed to scrutinize the core hypothesis.
    \newline \break
    \textbf{Advisor}: \\
    Marek Petrik


\end{document}

