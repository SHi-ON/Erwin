%%%%%%%% ICML 2020 EXAMPLE LATEX SUBMISSION FILE %%%%%%%%%%%%%%%%%

\documentclass{article}

% Recommended, but optional, packages for figures and better typesetting:
\usepackage{microtype}
\usepackage{graphicx}
\usepackage{subfigure}
\usepackage{booktabs} % for professional tables

% hyperref makes hyperlinks in the resulting PDF.
% If your build breaks (sometimes temporarily if a hyperlink spans a page)
% please comment out the following usepackage line and replace
% \usepackage{icml2020} with \usepackage[nohyperref]{icml2020} above.
\usepackage{hyperref}

% Attempt to make hyperref and algorithmic work together better:
\newcommand{\theHalgorithm}{\arabic{algorithm}}

% Use the following line for the initial blind version submitted for review:
%\usepackage{icml2020}

% If accepted, instead use the following line for the camera-ready submission:
\usepackage[accepted]{icml2020}
%\usepackage[preprint]{icml2020}


% SHi-ON packages
% argmin argmax begin
\usepackage{amsmath}
\usepackage{amsfonts}
\usepackage{xcolor}
\DeclareMathOperator*{\argmax}{arg\,max}
\DeclareMathOperator*{\argmin}{arg\,min}


% The \icmltitle you define below is probably too long as a header.
% Therefore, a short form for the running title is supplied here:
\icmltitlerunning{Submission and Formatting Instructions for ICML 2020}

\begin{document}

\twocolumn[
\icmltitle{ Density-based [Distributional] State Aggregation \\ in Reinforcement Learning}

% It is OKAY to include author information, even for blind
% submissions: the style file will automatically remove it for you
% unless you've provided the [accepted] option to the icml2020
% package.

% List of affiliations: The first argument should be a (short)
% identifier you will use later to specify author affiliations
% Academic affiliations should list Department, University, City, Region, Country
% Industry affiliations should list Company, City, Region, Country

% You can specify symbols, otherwise they are numbered in order.
% Ideally, you should not use this facility. Affiliations will be numbered
% in order of appearance and this is the preferred way.
\icmlsetsymbol{equal}{*}

\begin{icmlauthorlist}
\icmlauthor{Authors}{equal,to}
%\icmlauthor{Bauiu C.~Yyyy}{equal,to,goo}
%\icmlauthor{Cieua Vvvvv}{goo}
%\icmlauthor{Iaesut Saoeu}{ed}
%\icmlauthor{Fiuea Rrrr}{to}
%\icmlauthor{Tateu H.~Yasehe}{ed,to,goo}
%\icmlauthor{Aaoeu Iasoh}{goo}
%\icmlauthor{Buiui Eueu}{ed}
%\icmlauthor{Aeuia Zzzz}{ed}
%\icmlauthor{Bieea C.~Yyyy}{to,goo}
%\icmlauthor{Teoau Xxxx}{ed}
%\icmlauthor{Eee Pppp}{ed}
\end{icmlauthorlist}

\icmlaffiliation{to}{Department of Computer Science, University of New Hampshire}
%\icmlaffiliation{goo}{Googol ShallowMind, New London, Michigan, USA}
%\icmlaffiliation{ed}{School of Computation, University of Edenborrow, Edenborrow, United Kingdom}

\icmlcorrespondingauthor{Authors}{c.vvvvv@googol.com}
%\icmlcorrespondingauthor{Eee Pppp}{ep@eden.co.uk}

% You may provide any keywords that you
% find helpful for describing your paper; these are used to populate
% the "keywords" metadata in the PDF but will not be shown in the document
\icmlkeywords{Machine Learning, ICML}

\vskip 0.3in
]

% this must go after the closing bracket ] following \twocolumn[ ...

% This command actually creates the footnote in the first column
% listing the affiliations and the copyright notice.
% The command takes one argument, which is text to display at the start of the footnote.
% The \icmlEqualContribution command is standard text for equal contribution.
% Remove it (just {}) if you do not need this facility.

%\printAffiliationsAndNotice{}  % leave blank if no need to mention equal contribution
\printAffiliationsAndNotice{\icmlEqualContribution} % otherwise use the standard text.


\begin{abstract}
    Reinforcement learning (RL) problems often incur large scale state spaces needing sizable amount
    of computation to process.
    State aggregation as an analytically-transparent mean approximates and yields lower-dimension state spaces.
%   We present a state aggregation algorithm and a learning algorithm based on the former to solve a stochastic
%   problem.
    We present a state aggregation algorithm method in Markov decision processes (MDP) inspired by histograms.
    Our approach explores a \textit{splitting policy} in direct relationship with inter-quartile range
    of collected samples.
    We then accompanied an experiment by which we show the practical promises of the approach to achieve superior
    results.
\end{abstract}


\section{Introduction}
\label{sec:intro}
    Real-world problems embody high-dimensional state spaces where in computational issues
    arise.
    Various approximation methods are proposed in the literature to reduce the resulted value function
    dimensionality \cite{Sutton1998, Busoniu2010, Francois-Lavet2018}.
    State aggregation is one of the simplest and most transparent state representation approximation
    method.

    State aggregation, at the core level, is to discretize together similar states to ultimately yield a
    lower-dimension MDP. The reduced MDP requires considerably less amount of computational iterations or samples to
    solve, which improves convergence to the .

    Although state aggregation may not generalize as well as neural network-based methods, the techniques has both
    analytical and practical advantages \cite{Lagoudakis2003}.
    In comparison, state aggregation takes less computational power to calculate representative features.
    Moreover, analysis and troubleshooting in a state aggregation process is more obvious than a black-box neural
    network.
    Approximation is a corner stone in all successful RL methods to hold a "good" mapping between a large state
    space to a much smaller one where in value function is computable or at least feasible.
    These "good" approximators can end up with "good" features to represent the projected state space.

    The cross-fertilization in using state aggregation and deep neural networks in solving immense state
    space problems in RL is also profound.
    While the newly emerged methods are growing at a staggering pace, a parallel area of research has been formed
    wherein a variety of techniques have been utilized to make these methods contravene their promised functionality.

    Compared to other kernel-based density estimators such as \citep{HongLim2019}, histograms steer clear of feature
    selection step which is more of an experimental process than a framework of rules to follow.
    Our histogram-based state abstraction model yield a state space model by collecting samples in a domain-agnostic
    way.

    Our contribution is a two-fold.
    We first introduce an algorithm for aggregation in MDPs, then we move on to show advantages of taking such
    aggregation method to work.
    Hereby, we explain the paper structure in brief.
    We first lay down the core concepts of the MDP and the aggregation framework in Sections \ref{sec:mdp} and
    \ref{sec:state-aggregation} then continue to describe the methodology in Section \ref{sec:method}.


\section{Related Work}
    \label{sec:related-work}
    Aggregation was extensively appeared throughout the operations and computations research literature
    \cite{Chatelin1982, Rogers1991, Douglas1993}.
    It then introduced to the optimization by an iterative approach in linear programming \cite{Mendelssohn1982}.
    On the end of this spectrum, dynamic programming (DP) leveraged state aggregation to reduce computational size of
    the problem \cite{Bean1987}, as well as using an extension to state aggregation, \textit{soft state
    aggregation}, in an approximate value iteration method \cite{Singh1995}, and general approximate DP frameworks
    \cite{Gordon1995, Tsitsiklis1996}.

    In the recent body of work, state aggregation has applications in conjunction to other concepts such as
    temporal aggregation, so called \textit{options} \cite{Ciosek2015}, \textit{bottleneck simulator}
    \cite{Serban2018}, and also in continuous space optimal control \cite{Zhong2011}.

    Error analyses and regret bounds incured by state aggregation methods are comprehensively discussed by
    \citet{VanRoy2006} and \citet{Petrik2014}.
    Near optimality criterion in aggregation is also investigated in \citet{Bernstein2008}.

    It is noteworthy to point out that, state aggregation exists in the literature under different co-hyponyms.
    


\section{Markov Decision Processes}
\label{sec:mdp}
    Reinforcement learning problems can be formulated a Markov Decision Processes (MDP)
    framework \cite{Puterman1994, Sutton1998}.
    MDPs facilitate handling sequential decision-making under uncertainty.
    Discrete discounted reward criterion is the center of focus for ease of exposition in this paper.
    For an environment and a decision maker, consider the tuple $<>$ state space $S$ and action space $A$ are defined,
respectively.
    The transition probabilities $P$ describe how likely is the next state to incur by taking a specific action
    $a$ by the agent in a certain state $s$ which will later receive reward $r$.
    For a taken action $a_i \in A$ at the state $s_i \in S$, the decision maker will be ended up in state $s_{i+1}
    \in S$ with a likelihood expressed by the transition probabilities $P : S \times A \times S \rightarrow [0, 1]$
    and wil receive reward $r_{i+1}$ from the rewards $R : S \times A \rightarrow \mathbb{R}$.

%    \begin{equation}
%        s_{i+1} = P(s_{i+1}|s_i, a_i)
%    \end{equation}
%
%    \begin{equation}
%        r_{i+1} = R(s_i, a_i)
%    \end{equation}


\section{State Aggregation}
\label{sec:state-aggregation}
    Value function approximation alleviates the computational hurdle of a large state-space problem.
    We first introduce an approximate value iteration method to solve an aggregate problem.
    State aggregation is an approximation approach to make a problem more tractable by a reduction in dimensionality in
    state space \footnote{aggregation in action space although is possible, has less significance comparatively.}.
    State aggregation, in particular, is a parametric feature-based approximation where the features are membership
    functions of 0-1 form.
    Aggregation consolidates similar states together, so called representative states, which allows the
    \textit{aggregate} problem to be solved by an exact method \cite{Bertsekas2019}
    The computed value function from the exact solution of the aggregate problem can be then used in the original
    problem.

    Now we formulate the aggregation by introducing the dynamics of the aggregate problem.
    The transition probabilities between two representative states:

    \begin{equation}
        \hat{P}_{s s^\prime}(a)=\sum_{j=1}^{n} P_{s j}(a) \phi_{j y},
    \end{equation}

    where $p_ {s j}$ is the transition probabilities from a representative state, $s$, to an original state, $j$
    and $\phi_{j s^\prime}$ is the aggregation probabilities from an original state, $j$, to a representative state,
    $s^\prime$.
    The rewards are defined as:

    \begin{equation}
        \hat{r}(s, a)=\sum_{j=1}^{n} P_{s j}(a) r(s, a, j),
    \end{equation}

    where $\hat{r}(s, a)$ is the received reward at state $s$ by taking action $a$.


    Aggregation then helps us to derive the value function.
    Approximated optimal value function is a weighted sum of optimal rewards and can be computed by
    Equation \ref{approx val func} where $r_{s^\prime}^{\ast}$ is optimal reward at represented state $s$.

    \begin{equation}
        \label{approx val func}
        \tilde{V}(j)=\sum_{s^\prime \in \mathcal{U}} \phi_{j s^\prime} r_{s^\prime}^{\ast},
    \end{equation}


\section{Method}
\label{sec:method}
    Splitting the observations into spans of similar observations is one way to interpret an aggregation
    problem.
    Histograms as classical and simple density estimators were around in the literature for decades \cite{Scott1979,
    Scott2015}.
    Inspired by histograms and the statistics behind finding the proper number of bins for a given distribution, we
    build our proposed aggregation method on top of such bases.
    Proper \textit{bin width}, according to the definition, is to find a trade-off to capture similarities in data
    features and diminish variations caused by random sampling \cite{Knuth2019}.

    \subsection{Bin width}
    Finding the bin width for a given distribution could be calculated by Freedman-Diaconis's choice
    \cite{Freedman1981}.
    As the rule states:

    \begin{equation}
        |S_{agg}| = 2 \frac{\operatorname{IQR}(x)}{\sqrt[3]{n}}
    \end{equation}

    where n is number of observations and $IQR(x)$ is the inter-quartile range of data.
    The segmentation rule exhibits robustness to sample distribution due to low sensitivity of IQR to outliers and
    heteroscedasticity in the given distribution compared to conventional dispersion measures such as variance or
    standard deviation.
    State values, in this paper, are considered as the similarity metric \textcolor{red}{[measure]} in state
    aggregation.

    By having the inter-quartile range of data for each feature, a discretization policy can be calculated.

    We developed an aggregation scheme to lower the state space dimensionality by assigning same action to the state
    in a neighbourhood, so called nearest neighbour/ piecewise linear aggregation.
    We then solve directly the lower dimension problem which relatively takes less computation.

    The aggregate problem is stochastic even if the original problem is deterministic.
    Once we lay out the aggregate problem framework we can solve the problem by any exact methods either value- or
    policy-space solutions.

    The objective is finding an optimal aggregate model which falls in the span of spectrum from a fine-grid model
    to a rigid aggregate model.
    Our presented method calculates a discretization policy for which we calculate a feasible value function.
    Return of a policy shows the quality of the gauged policy.
    Hence, by calculating the accumulative return (eq.\ref{eq:return}) over run of the \textit{true} model and in a
    comparison with the original return we can measure the improvement.

    \begin{equation}
        R = \sum_{k=0}^{\infty} \gamma^{k} r_{k+1}. \label{eq:return}
    \end{equation}

    \subsection{Aggregation}
    The original states with the closest \textit{calculated} values forgather to build aggregate states.
    For this purpose, state values are clustered by K-means analysis to calculate the relevant aggregate states.
    State aggregation merge similar transitions based on a triple $<s, s', a>$ where $s$ indicates the starting state,
    $s'$ is the ending state, and the take action in that transition represented by $a$.
    In the process of stacking similar transitions, we here presume all inbound transitions as outbound transition of
    the ending state.
    This assumption render all transitions as outbound ones, therefore we assign the average probability as the
    transition probability in the aggregate model.


\section{Experiment}
    We apply the approach to two MDPs proposed by \cite{Strehl2004}.
    \textit{RiverSwim} an MDP consists of six states and \textit{SixArms} one with seven states.
    In \textit{RiverSwim} flow of the river is to left and the swimmer picks probabilistically equal either state 1
    or 2 as the start point.
    All states but two terminal states have zero reward to land at while the right most one rewards substantially
    higher than the other one.
    The agent in SixArms selects between six distinct actions which pulls different arms of a multi-armed bandit.
    By pulling an arm the agent traverses to a new state.
    Although the agent does not obtain reward by pulling the arms but going to the connected state to that arm is
    highly rewarding in which the transition probability is in inverse relationship with the reward value.

    As both problems favor keeping the agent in states with lower reward, exploration is paramount to maximize the
    accumulated reward.
    This is also confirmed by the authors in \cite{Strehl2004} that the agent will fail to learn when a rudimentary
    $\epsilon$-greedy algorithm.

    We use a randomized policy to simulate trajectories and collect samples to attenuate bias in action selection
    process.
    As it has shown the literature, $\epsilon$-greedy has a bias in solving exploration-exploitation dilemma, we
    explore the environment by following a randomized policy with a uniform distribution.

    \subsection{Machine Replacement}
    This domain as a \textit{non-episodic} (continuous) task maintains a set of machines.
    The goal is to find a policy to incur the least repair needed.
    incure


\section{Future Works}
    We developed the presented approach with having reproducibility in mind.
    It could be served as a launching pad for the future contributors who want to surf this newly emerged field deeper.
    Future ideas to pursue is a generative model to be able to generate adversaries from a network.


\bibliography{library}
\bibliographystyle{icml2020}


\end{document}